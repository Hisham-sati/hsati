%!TEX root = ../everything.tex

{\color{red}\textbf{TODO!} talk about Julien's thesis etc and why the following things are what we need to show}

\section{Axiomatic Chern classes} % (fold)
\label{sec:axiomatic_chern_classes}

    \emph{Throughout this section, let $X$ be a paracompact complex-analytic manifold with a `nice' (as in \cref{sec:preliminaries}) cover $\cover$.}

    \subsection{Line bundles} % (fold)
    \label{sub:line_bundles}

        Let $\mathcal{L}$ be a line bundle on $X$, defined by transition maps $\{g_{\alpha\beta}\}_{\alpha,\beta}$.
        The first Chern class is, classically,
        \begin{equation}\label{equation:first-chern-class-of-line-bundle}
            c_1(\mathcal{L}) = \left\{\frac{\d g_{\alpha\beta}}{g_{\alpha\beta}}\right\}_{\alpha,\beta}\in\check\H(\cover,\Omega_X^1).
        \end{equation}
        Comparing this with \cref{equation:first-simplicial-atiyah-class-explicit} we see that the trace of the first (simplicial) Atiyah class agrees with the first Chern class for line bundles.
    
    % subsection line_bundles (end)

    \subsection{Whitney sum formula} % (fold)
    \label{sub:whitney_sum_formula}

        {\color{red}\textbf{TODO:} change this to the \emph{total} Chern class, or Chern characters}

        \begin{lemma}[Whitney sum formula]
            {\color{red}\textbf{should be for SESs}}
            Let $\E_1$ and $\E_2$ be vector bundles on $X$.
            Then, for all $k\in\mathbb{N}$, the Whitney sum formula holds:
            \begin{equation*}
                \underline{\at}^k_{\E_1\oplus\E_2} = \sum_{i+j=k}\underline{\at}^i_{\E_1}\underline{\at}^j_{\E_2}.\qedhere
            \end{equation*}
        \end{lemma}

        \begin{proof}
            {\color{red}\textbf{TODO!}}
        \end{proof}
    
    % subsection whitney_sum_formula (end)

    \subsection{Pullbacks} % (fold)
    \label{sub:pullbacks}

        \begin{lemma}[Functoriality under pullbacks]
            Let $f\colon Y\to X$ be holomorphic (where $Y$ is paracompact, and has a cover $\mathcal{V}$ satisfying the same properties as $\cover$), and let $\E$ be a vector bundle on $X$.
            Then, for all $k\in\mathbb{N}$,
            \begin{equation*}
                f^*(\underline{\at}^k_\E) = \underline{\at}^k_{(f^*\E)}.\qedhere
            \end{equation*}
        \end{lemma}

        \begin{proof}
            {\color{red}\textbf{TODO! -- need to talk about derived pullbacks (and flatness of multiplication of an SES by a scalar?)}}
        \end{proof}
    
    % subsection pullbacks (end)

    \subsection{The splitting principle} % (fold)
    \label{sub:the_splitting_principle}

        {\color{red}\textbf{$\mathbb{P}^1$-homotopy invariance of cohomology theory}}

        \begin{theorem}
            Let $\E$ be a vector bundle on $X$.
            Then the trace of the $k$-th Atiyah class is the $k$-th Chern class:
            \begin{equation*}
                \tr\int_{|\Delta|}\underline{\at}_\E^k \cong \operatorname{c}_k(\E) \in \H^{k}(\Omega_X^{\geqslant k}) {\color{red}???}.\qedhere
            \end{equation*}
        \end{theorem}

        \begin{proof}
            {\color{red}\textbf{splitting principle}}
            {\color{red}\textbf{TODO!}}
        \end{proof}
    
    % subsection the_splitting_principle (end)

    \subsection{Simplicial Atiyah classes} % (fold)
    \label{sub:simplicial_atiyah_classes}
    
    % subsection simplicial_atiyah_classes (end)

% section axiomatic_chern_classes (end)

\section{Transgression} % (fold)
    \label{sec:transgression}

    This will be used in \cref{prt:vector_bundles_deligne}.

% section transgression (end)

