%!TEX root = ../everything.tex

        \subsection{History and comparisons} % (fold)
        \label{sub:history_and_comparisons}

            {\color{red}\textbf{what do we do that others don't?}}
        
        % subsection history_and_comparisons (end)
    
    {\color{red}\textbf{PROOF READ ALL OF THIS PART}}
    {\color{red}\textbf{basically want a section-by-section description}}

        \subsection*{Part I} % (fold)
        \label{sub:part_i}
        
            {\color{red}\cite{Green:1980wpa} and \cite{Dupont:1976up} work for smooths things; we `skip' the smooth stuff and go straight to holomorphic forms with smooth parameters (whence another version of dupont's fibre integration lemma)}

            {\color{red}In fact, this is where the power of the simplicial construction is hiding: in the smooth case we can always construct global connections, whereas we cannot always do so in the holomorphic case. By working with forms over $|\Delta_p|\times X^p_\cover$ we are mixing in `just enough' smooth data to be able to construct global connections. This is explained in \cref{sec:simplicial_construction}.}

        % subsection part_i (end)

        \subsection{Part II} % (fold)
        \label{sub:part_ii}

            !
        
        % subsection part_ii (end)

        \subsection*{Part III} % (fold)
        \label{sub:part_iii}

            {\color{red}\textbf{TODO}}

            {\color{red}\textbf{historical overview (nothing new here!)}}

            \cref{sec:holomorphic-twisting-cochains,sec:dg-categories}



            {\color{red}!!!!!!!!!!!!!!!!}This section is largely just notes on, and summaries of, important papers in the literature of twisting cochains.

            No historical account is ever complete.
            In particular, we don't really mention the `first' reference to twisting cochains: \cite{Brown:1959ia}; we also don't follow what happened to the subject when it branched off into differential homological algebra (namely \cite{Moore:70dh}), even though this also predates all of the material that we \textit{do} cover.\footnote{For a good summary of the subject from a differential and lie-algebraic viewpoint, see \cite{Stasheff:2009cc}.}
            Our focus is really split into two parts: firstly, the development of twisting cochains by Toledo and Tong (and O'Brian) in \cite{Toledo:1976gy,Toledo:1978tq,OBrian:1981vs} using Čech cohomology and explicit methods; secondly, the development of twisted complexes (from \cite{Bondal:1991un}\footnote{In some papers this is cited as \textit{Framed triangulated categories} instead of \textit{Enhanced triangulated categories}, but this is just an artefact of translation from the original paper (which is in Russian).}) and the application of the language of DG-categories in \cite{Block:2015vk,Wei:2016tv,Wei:2016ip}.
            We discuss only briefly some of the generalisations to $A_\infty$-categories, such as \cite{Faonte:2015vc}.

            Here we give the second of our two historical overviews on twisting cochains.
            This is much shorter than the first, and only briefly describes the modern viewpoint of twisting cochains as twisted complexes.{\color{red}!!!!!!!!!!!}
        
        % subsection part_iii (end)

        \subsection*{Part IV} % (fold)
        \label{sub:part_iv}

            {\color{red}\textbf{TODO}}

            {\color{red}reviewing green's construction; adapting it to our framework}
        
        % subsection part_iv (end)
