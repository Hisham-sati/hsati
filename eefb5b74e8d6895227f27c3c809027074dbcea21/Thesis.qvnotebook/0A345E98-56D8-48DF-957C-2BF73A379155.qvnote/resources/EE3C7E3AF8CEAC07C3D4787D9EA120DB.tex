%!TEX root = ../everything.tex

\section{The story} % (fold)
\label{sec:the_story}

    \begin{enumerate}[(i)]
        \item twisting cochains resolve coherent sheaves
        \begin{enumerate}
            \item Green's resolution is actually a strict resolution: we use the twisting cochains (up to homotopy) to construct true simplicial vector bundles; we get a concrete way of computing things {\color{red}\textbf{question:} is this resolution cofibrant?}
            \item after this, we forget completely about twisting cochains
        \end{enumerate}
        \item Green's construction (plus Dupont's fibre integration) gives us classes in Hodge cohomology for simplicial vector bundles
        \item Julien's thesis/paper tells us what we need to show these classes satisfy to know that they are Chern classes (since he's already proved that Hodge cohomology satisfies nice enough properties)
        \begin{enumerate}
            \item \emph{agrees on line bundles:} this is a straightforward calculation
            \item \emph{functorial under pullbacks:} N.B. for coherent sheaves, we get that the Chern class of the derived pullback is the pullback of the Chern class, but using Green's resolution we can work with the usual pullback to calculate the derived one, i.e. we just need to show that the Chern class of simplicial vector bundles is functorial under the true pullback
            \item \emph{Whitney sum for short exact sequences:} using \cref{lemma:split-exact-suffices}, it suffices to show this for split exact sequences of coherent sheaves, but this is in Green
            \item \emph{Riemann-Roch for closed immersions:} apparently this follows from the other three properties by ``deformations to the normal cone'' or some such algebraic geometry magic
        \end{enumerate}
    \end{enumerate}

    \begin{lemma}\label{lemma:split-exact-suffices}
        If Chern classes are additive on every split exact sequence of coherent sheaves then they are additive on every short exact sequence of coherent sheaves.
    \end{lemma}

    \begin{proof}
        Let $0\to\mathscr{F}\xrightarrow{\iota}\mathscr{G}\xrightarrow{\pi}\mathscr{H}\to0$ be a short exact sequence of coherent sheaves on $X$, and $t\in\mathbb{C}$ (which can be thought of as $t\in\Gamma(\mathbb{C},\mathbb{P}^1)$).
        Write $p\colon X\times\mathbb{P}^1\to X$ to mean the projection map.
        Define
        \begin{equation*}
            \mathscr{N} = \Ker\big(p^*\mathscr{G}(1)\oplus p^*\mathscr{H}\xrightarrow{\pi(1)-t\cdot\id}p^*\mathscr{H}(1)\big)
        \end{equation*}
        where $(\pi(1)-t\cdot\id)\colon (g\otimes y, h)\mapsto \pi(g)\otimes y - h\otimes t$.
        We claim that this gives a short exact sequence
        \begin{equation}\label{equation:ses-p1-inv}
            0\to p^*\mathscr{F}(1)\to\mathscr{N}\to p^*\mathscr{H}\to0
        \end{equation}
        of sheaves over $X\times\mathbb{P}^1$.
        where the maps are the `obvious' ones: $p^*\mathscr{F}(1)\to\mathscr{N}$ is the map $\iota(1)\colon p^*\mathscr{F}(1)\to p^*\mathscr{G}(1)$ included into $p^*\mathscr{G}(1)\oplus p^*\mathscr{H}$ (which we prove lands in $\mathscr{N}$ below); and $\mathscr{N}\to p^*\mathscr{H}$ is the projection $p^*\mathscr{G}(1)\oplus p^*\mathscr{H}\to p^*\mathscr{H}$ restricted to $\mathscr{N}$
        
        \begin{itemize}
            \item To prove surjectivity, let $h\in\Gamma(U,p^*\mathscr{H})$.
                Then $h\otimes t\in\Gamma(U,p^*\mathscr{H}(1))$.
                But $\pi\colon\mathscr{G}\to\mathscr{H}$ is surjective, and thus so too is the induced map $\pi(1)\colon p^*\mathscr{G}(1)\to p^*\mathscr{H}(1)$, hence there exists $g\otimes y\in\Gamma(U,p^*\mathscr{G}(1))$ such that $\pi(g\otimes y)=h\otimes t$.
                Thus $(g\otimes y, h)\in\mathscr{N}$ maps to $h$.
            \item To prove injectivity (and that this map is indeed well defined), we use the fact that tensoring with $\mathscr{O}(1)$ is exact, and so, in particular, $\iota(1)\colon p^*\mathscr{F}(1)\to p^*\mathscr{G}(1)$ is injective.
                The inclusion into the direct sum $p^*\mathscr{G}(1)\oplus p^*\mathscr{H}$ is injective by the definition of a direct sum, so all that remains to show is that the image of this composite map is contained inside $\mathscr{N}$.
                Let $f\otimes x\in\Gamma(U,p^*\mathscr{F}(1))$.
                Then this maps to $(\iota(f)\otimes x,0)\in p^*\mathscr{G}(1)\oplus p^*\mathscr{H}$, but this is clearly in the kernel of $\pi(1)-t\cdot\id$ since $\pi\iota(f)=0$.
            \item To prove exactness, it suffices to show that $\Ker(\mathscr{N}\to p^*\mathscr{F})\cong p^*\mathscr{F}(1)$.
                But
                \begin{align*}
                    \Ker(\mathscr{N}\to p^*\mathscr{F}) &= \{(g\otimes y, h)\in\mathscr{N} \mid h=0\}\\
                    &= \{(g\otimes y, h)\in p^*\mathscr{G}(1)\oplus p^*\mathscr{H} \mid h=0\text{ and }\pi(g)\otimes y-h\otimes t=0\}\\
                    &= \{(g\otimes y, h)\in p^*\mathscr{G}(1)\oplus p^*\mathscr{H} \mid \pi(g)\otimes y=0\}\\
                    &= \{(g\otimes y, h)\in p^*\mathscr{G}(1)\oplus p^*\mathscr{H} \mid (g\otimes y)\in\Im\iota(1)\}\\
                    &\cong p^*\mathscr{F}(1).
                \end{align*}
        \end{itemize}
        
        Now we claim that the short exact sequence \eqref{equation:ses-p1-inv} is split for $t=0$, and has $\mathscr{N}\cong p^*\mathscr{G}$ for $t\neq0$.
        Formally, we do this by looking at the pullback of the map $X\times\{t\}\to X\times\mathbb{P}^1$, but for the moment we just `pick a value for $t$'.

        \begin{itemize}
            \item $t=0$.
                By definition,
                \begin{align*}
                    \mathscr{N} &= \Ker\big(p^*\mathscr{G}(1)\oplus p^*\mathscr{H}\xrightarrow{\pi(1)-t\cdot\id}p^*\mathscr{H}(1)\big)\\
                    &= \Ker\big(p^*\mathscr{G}(1)\oplus p^*\mathscr{H}\xrightarrow{\pi(1)}p^*\mathscr{H}(1)\big)\\
                    &\cong \Ker\big(p^*\mathscr{G}(1)\xrightarrow{\pi(1)}p^*\mathscr{H}(1)\big) \oplus p^*\mathscr{H}\\
                    &\cong p^*\mathscr{F}(1)\oplus p^*\mathscr{H}.
                \end{align*}
            \item $t\neq0$.
                Define the injective morphism $\varphi\colon p^*\mathscr{G}\to\mathscr{N}$ of coherent sheaves by $\varphi(g)=(g\otimes t,\pi(g))$.
                To see that this is also surjective, let $(g\otimes y, h)\in\mathscr{N}$.
                If $y=0$ then we must have $h=0$, and so $(g\otimes y, h) = (0,0) = \varphi(0\otimes0)$.
                If $y\neq0$ then $\pi(g)\otimes y-h\otimes t=0$, with $y,t\neq0$, whence $\pi(g)=\frac{y}{t}h$.
                Then $(g\otimes y, h) = (\frac{t}{y}g\otimes t, \pi(\frac{t}{y}g)) = \varphi(\frac{t}{y}g).$
        \end{itemize}

        As one final ingredient, note that any coherent sheaf on $X$ pulled back to a sheaf on $X\times\mathbb{P}^1$ is flat over $\mathbb{P}^1$, and so $\mathscr{N}$ is flat over $\mathbb{P}^1$, since both $\mathscr{F}(1)$ and $\mathscr{H}$ are.
        Thus, for $\tau_t\colon X\times\{t\}\to X\times\mathbb{P}^1$ given by a choice of $t\in\mathbb{C}$, the derived pullback $\mathbb{L}\tau_t^*\mathscr{N}$ agrees with the usual pullback $\tau_t^*\mathscr{N}$.

        Now we use the $\mathbb{P}^1$-homotopy invariance of de Rham cohomology: the induced map
        \begin{equation*}
            \tau_t^*\colon\H^\bullet\big(X\times\mathbb{P}^1,\Omega_{X\times\mathbb{P}^1}^\bullet\big) \to \H^\bullet\big(X\times\{t\},\Omega_{X\times\{t\}}^\bullet\big)
        \end{equation*}
        doesn't depend on the choice of $t$.
        Since $X\times\{t\}$ is (canonically) homotopic to $X$, we can identify $(p\tau_t)^*$ with the identity on $\H^\bullet(X,\Omega_X^\bullet)$.
        Since \cref{equation:ses-p1-inv} splits for $t=0$, by our hypothesis, flatness, and the fact that Green's construction is functorial under derived pullback, we know that
        \begin{align*}
            \tau_0^*c(\mathscr{N}) = c(\mathbb{L}\tau_0^*\mathscr{N}) = c(\tau_0^*\mathscr{N}) &= c(\tau_0^*p^*\mathscr{F}(1)\oplus\tau_0^*p^*\mathscr{H})\\
            &= c(\mathscr{F})\wedge c(\mathscr{H}).
        \end{align*}
        But we also know that $\mathscr{N}\cong p^*\mathscr{G}$ for $t\neq0$, and so
        \begin{equation*}
            c(\mathscr{G}) = (p\tau_t)^*c(\mathscr{G}) = \tau_t^*c(\mathscr{N}).
        \end{equation*}
        So, finally, the $t$-invariance of $\tau^*$ tells us that
        \begin{equation*}
            c(\mathscr{G}) = c(\mathscr{F})\wedge c(\mathscr{H}).\qedhere
        \end{equation*}
    \end{proof}

% section the_story (end)

\section{Preliminaries} % (fold)
\label{sec:preliminaries}

    Throughout, let $\E$ be a vector bundle of rank $\mathfrak{r}$ on $X$, where $(X,\O_X)$ is a (paracompact) complex-analytic manifold, with a `sufficiently nice'\footnote{That is, locally finite and Stein, and such that $\E$ is free over each $U_\alpha$} open cover $\cover=\{U_\alpha\}_{\alpha\in I}$.
    For all $\alpha\in I$, let $\nabla_\alpha$ be some flat\footnote{These connections aren't required to be flat in order for \cref{lemma:atiyah-class-difference-of-connections} to hold, but we make this assumption now anyway. For example, we could simply pick the trivial connection $\d$ over each $U_\alpha$.} connection over $U_\alpha$.

    The isomorphisms $\varphi_\alpha\colon\E|_{U_\alpha}\congto(\O_X|_{U_\alpha})^\mathfrak{r}$ define
    \begin{equation}
        M_{\alpha\beta} = \varphi_\alpha\circ\varphi_\beta^{-1}\colon (\O_X|_{U_\beta})^\mathfrak{r}\congto(\O_X|_{U_\alpha})^\mathfrak{r}.
    \end{equation}
    So picking a basis of $\nabla_\alpha$-flat sections $\{s^{\alpha}_1,\ldots,s^{\alpha}_\mathfrak{r}\}$ over $U_\alpha$ means that the $M_{\alpha\beta}$ can be realised as explicit $(\mathfrak{r}\times \mathfrak{r})$-matrices that describe the base change from the trivialisation over $U_\alpha$ to the trivialisation over $U_\beta$:
    \begin{equation}
        s^\alpha_k = \sum_\ell(M_{\alpha\beta})_k^\ell s^\beta_\ell
    \end{equation}

    \subsection{The Atiyah class} % (fold)t
    \label{sub:the_atiyah_class}
    
        \begin{definition}[Atiyah exact sequence]
            The \emph{Atiyah exact sequence of $\E$} is the short exact sequence of $\O_X$-modules
            \[
                0 \to \E\otimes\Omega_X^1 \to J^1(\E) \to \E \to 0
            \]
            where $J^1(\E) = (\E\otimes\Omega_X^1)\oplus\E$ as a $\mathbb{C}_X$-module\footnote{Where $\mathbb{C}_X$ is the constant sheaf.} but we define the $\O_X$-action by
            \[
                f(s\otimes\omega,t) = (fs\otimes\omega+t\otimes\d f, ft).\qedhere
            \]
        \end{definition}

        By the above definition, a holomorphic connection on $\E$ is exactly a splitting of the Atiyah exact sequence of $\E$.

        \begin{note}\label{note:iterated-connection}
            Recall that we can extend any connection $\nabla\colon\mathcal{F}\to\mathcal{F}\otimes\Omega^1$ to a map $\nabla\colon\mathcal{F}\otimes\Omega^r\to\mathcal{F}\otimes\Omega^{r+1}$ by enforcing the Leibnitz rule:
            \[
                \nabla(s\otimes\omega) = \nabla(s)\wedge\omega + s\otimes\d\omega.\qedhere
            \]
        \end{note}

        \begin{definition}[Atiyah class]
            The \emph{Atiyah class $\at_\E$ of $\E$} is the class of $J^1(\E)$ in the first $\Ext$ group:
            \[
                \at_\E = [J^1(\E)] \in \Ext_{\O_X}^1(\E,\E\otimes\Omega_X^1)\qedhere
            \]
        \end{definition}

        \begin{lemma}\label{lemma:atiyah-class-difference-of-connections}
            The Atiyah class of $\E$ is represented by the cocycle\footnote{We omit the restriction from our notation: really we mean $\nabla_\alpha|_{U_{\alpha\beta}}-\nabla_\beta|_{U_{\alpha\beta}}$}
            \[
                \{\nabla_\beta-\nabla_\alpha\}_{\alpha,\beta\in I} \in\cech^1\big(\sheafhom(\E,\E\otimes\Omega_X^1)\big).\qedhere
            \]
        \end{lemma}

        \begin{proof}
            First, recall that the difference of any two connections is simply an $\O_X$-linear map.
            Secondly, note that we do indeed have a cocycle:
            \[
                (\nabla_\beta-\nabla_\alpha) + (\nabla_\gamma-\nabla_\beta) = \nabla_\gamma-\nabla_\alpha.
            \]
            Thus $\{\nabla_\beta-\nabla_\alpha\}_{\alpha,\beta}\in\cechnou^1\big(X,\sheafhom(\E,\E\otimes\Omega_X^1)\big)$.
            Then we use the isomorphisms
            \[
                \Ext_{\O_X}^1(\E,\E\otimes\Omega_X^1) \cong \Hom_{\mathcal{D}(X)}(\E,\E\otimes\Omega_X^1[1]) \cong \H^1\big(X,\sheafhom(\E,\E\otimes\Omega_X^1)\big).
            \]
            Finally, we have to prove that this class defined in homology agrees with that defined in our definition of the Atiyah class of $\E$.
            This fact is true in more generality, and we prove it as so.

            Let $0\to \mathcal{A}\to \mathcal{B}\to \mathcal{C}\to 0$ be a short exact sequence in some abelian category $\mathscr{C}$.
            The definition of $[\mathcal{B}]\in\Ext_\mathscr{C}^1(\mathcal{C},\mathcal{A})$ is as the class in $\Hom_{\mathcal{D}(\mathscr{C})}(\mathcal{C},\mathcal{A}[1])$ of a canonical morphism $\mathcal{C}\to\mathcal{A}[1]$ constructed using $\mathcal{B}$ as follows:
            \begin{enumerate}[(i)]
                \item take the quasi-isomorphism $(\mathcal{A}\to \mathcal{B})\congto \mathcal{C}$, where $\mathcal{B}$ is in degree $0$;
                \item invert it to get a map $\mathcal{C}\congto(\mathcal{A}\to \mathcal{B})$ such that the composite $\mathcal{C}\congto(\mathcal{A}\to \mathcal{B})\to\mathcal{B}\to\mathcal{C}$ is the identity;
                \item compose with the identity map $(\mathcal{A}\to \mathcal{B})\to \mathcal{A}[1]$.
            \end{enumerate}
            In the case of locally-free sheaves on $X$ we can realise the quasi-isomorphism $\mathcal{C}\congto(\mathcal{A}\to \mathcal{B})$ as $\mathcal{C}\congto\cech^\bullet(\mathcal{A}\to\mathcal{B})$, using the Čech complex of a complex:
            \[
                \cech^\bullet(\mathcal{A}\to \mathcal{B}) = \cech^0(\mathcal{A}) \xrightarrow{(\check\delta,f)} \cech^1(\mathcal{A})\oplus\cech^0(\mathcal{B}) \xrightarrow{(\check\delta,-f,\check\delta)} \cech^2(\mathcal{A})\oplus\cech^1(\mathcal{B}) \xrightarrow{(\check\delta,f,\check\delta)} \ldots
            \]
            where $\cech^0(\mathcal{A})$ is in degree $-1$.
            If we have local sections $\sigma_\alpha\colon \mathcal{C}|_{U_\alpha}\to \mathcal{B}|_{U_\alpha}$ then $\sigma_\beta-\sigma_\alpha$ lies in the kernel $\Ker(\mathcal{B}|_{U_{\alpha\beta}}\to \mathcal{C}|_{U_{\alpha\beta}})$, and so we can lift this difference to $\mathcal{A}$, giving us the map
            \[
                (\{\sigma_\alpha\}_\alpha, \{\sigma_\beta-\sigma_\alpha\}_{\alpha,\beta})\colon \mathcal{C} \to \cech^0(\mathcal{B})\oplus\cech^1(\mathcal{A}).
            \]
            This map we have constructed is exactly $[B]$.
            More precisely,
            \[
                \begin{array}{rcccl}
                    \Ext_{\O_X}^1(\mathcal{C},\mathcal{A}) &\cong& \Hom_{\mathcal{D}(X)}(\mathcal{C},\mathcal{A}[1]) &\cong& \H^1\big(X,\sheafhom(\mathcal{C},\mathcal{A})\big)\\[.8em]
                    {[B]} &\leftrightarrow& \mathcal{C}\congto(A\to B)\to A[1] &\leftrightarrow& [\{\sigma_\alpha-\sigma_\beta\}_{\alpha,\beta}].
                \end{array}
                \vspace{-1.5em}
            \]
        \end{proof}

        \begin{note}\label{note:endomorphism-valued-forms}
            When $\mathcal{F}$, $\mathcal{G}$, and $\mathcal{H}$ are sheaves of $\O_X$-modules, with $\mathcal{H}$ locally free, we have the isomorphism
            \[
                \sheafhom(\mathcal{F},\mathcal{G}\otimes\mathcal{H}) \cong \sheafhom(\mathcal{F},\mathcal{G})\otimes\mathcal{H}.
            \]
            This means that, taking the trivialisation over $U_\alpha$, we can consider $\omega_{\alpha\beta}=(\nabla_\beta-\nabla_\alpha)$ as an $\mathfrak{r}\times \mathfrak{r}$-matrix of $1$-forms on $X$, where $\mathfrak{r}=\operatorname{rank}\E$, since
            \[
                \H^1\big(X,\sheafhom(\E,\E\otimes\Omega_X^1)\big) \cong \H^1\big(X,\Omega_X^1\big(\sheafend(\E)\big)\big)
            \]
            where $\Omega_X^r(\mathcal{F})=\Omega_X^r\otimes\mathcal{F}$ is the collection of $r$-forms on $X$ with values in $\mathcal{F}$.
            We calculate $\omega_{\alpha\beta}$ explicitly in \cref{sub:the_first_atiyah_class}.
        \end{note}

        \begin{definition}
            We write $\omega_{\alpha\beta}$ to mean $(\nabla_\beta-\nabla_\alpha)$ considered as an $\mathfrak{r}\times \mathfrak{r}$-matrix of $1$-forms on $X$, as in \cref{note:endomorphism-valued-forms}.
            We calculate $\omega_{\alpha\beta}$ explicitly in \cref{equation:nabla-minus-nabla-omega}.
        \end{definition}

        Recall that for sheaves $\mathscr{F},\mathscr{G}$ of $\mathcal{O}_X$-modules we have the cup product
        \[
            \smile \colon \H^m(X,\mathscr{F})\otimes\H^n(X,\mathscr{G}) \to \H^{m+n}(X,\mathscr{F}\otimes\mathscr{G})
        \]
        which is given in Čech cohomology by the tensor product: $(a\smile b)_{ijk} = (a)_{ij}\otimes(b)_{jk}$.

        \begin{example}[Formal construction of the second Atiyah class]\label{example:second-atiyah-class-formal}
            Take
            \[
                (\at_\E\otimes\,\id_{\Omega_X^1})\smile(\at_\E) \in \H^2\big(X,\sheafhom(\E\otimes\Omega_X^1,\E\otimes\Omega_X^1\otimes\Omega_X^1)\otimes\sheafhom(\E,\E\otimes\Omega_X^1)\big)
            \]
            and apply the composition map
            \[
                \H^m\big(X,\sheafhom(\mathcal{G},\mathcal{H})\otimes\sheafhom(\mathcal{F},\mathcal{G})\big) \to \H^m\big(X,\sheafhom(\mathcal{F},\mathcal{H})\big)
            \]
            to obtain
            \[
                (\at_\E\otimes\,\id_{\Omega_X^1})\smile(\at_\E) \in \H^2\big(X,\sheafhom(\E,\E\otimes\Omega_X^1\otimes\Omega_X^1)\big) \cong \H^2\big(X,\sheafend(\E)\otimes\Omega_X^1\otimes\Omega_X^1)\big).
            \]
            Finally, apply the wedge product to get
            \[
                (\at_\E\otimes\,\id_{\Omega_X^1})\wedge(\at_\E) \in \H^2\big(X,\sheafend(\E)\otimes\Omega_X^2)\big).\qedhere
            \]
        \end{example}

        \textbf{\emph{THIS CONSTRUCTION GIVES US THE EXPONENTIAL CHERN CLASSES. TO OBTAIN THE USUAL ONES, YOU NEED TO WEDGE THE ENDOMORPHISM PART, NOT THE FORM PART.}}

        \begin{definition}[Higher Atiyah classes]\label{definition:higher-atiyah-classes}
            The \textit{$p$-th Atiyah class} is the class
            \[
                \at_\E^p = (\at_\E\otimes\id_{\Omega_X^1}^{\otimes(p-1)})\wedge\ldots\wedge(\at_\E\otimes\,\id_{\Omega_X^1})\smile(\at_\E) \in \H^p\big(X,\sheafend(\E)\otimes\Omega_X^p)\big)
            \]
            where we apply composition and the wedge product as in \cref{example:second-atiyah-class-formal}.
        \end{definition}
        
        \begin{example}[Local expression for the second Atiyah class]\label{example:second-atiyah-class}
            We can find an explicit representative for the second Atiyah class by using \cref{note:endomorphism-valued-forms}:
            \begin{align*}
                \big((\at_\E\otimes\id_{\Omega_X^1})\smile(\at_\E)\big)_{\alpha\beta\gamma} &= (\at_\E\otimes\id_{\Omega_X^1})_{\alpha\beta}\otimes(\at_\E)_{\beta\gamma}\\
                &\leftrightarrow (\omega_{\alpha\beta}\otimes\id_{\Omega_{U_{\alpha\beta}}^1})\otimes\omega_{\beta\gamma} \in \mathcal{M}_\mathfrak{r}(\Omega_{U_{\alpha\beta}}^1\otimes\Omega_{U_{\alpha\beta}}^1) \otimes \mathcal{M}_\mathfrak{r}(\Omega_{U_{\beta\gamma}}^1)
            \end{align*}
            where $\mathcal{M}_\mathfrak{r}(A)$ is the collection of $A$-valued $(\mathfrak{r}\times \mathfrak{r})$-matrices.
            Now, before composing these two matrices, as described in \cref{example:second-atiyah-class-formal,definition:higher-atiyah-classes}, we first have to account for the change of trivialisation $U_{\beta\gamma}\to U_{\alpha\beta}$.
            That is, after applying composition and the wedge product, we have
            \[
                (\at_\E^2)_{\alpha\beta\gamma} = [\omega_{\alpha\beta}\wedge M_{\alpha\beta}\omega_{\beta\gamma}M_{\alpha\beta}^{-1}].\qedhere
            \]
        \end{example}

        \begin{example}[Local expressions for higher Atiyah classes]
            We know that $\at_\E^3$ is represented locally by $\omega_{\alpha\beta}\omega_{\beta\gamma}\omega_{\gamma\delta}$, but where $\omega_{\beta\gamma}$ and $\omega_{\gamma\delta}$ undergo a base change to become $\Omega_{U_{\alpha\beta}}^1$-valued.
            But then, do we
            \begin{enumerate}[(i)]
                \item base change $\omega_{\gamma\delta}$ to an $\Omega_{U_{\beta\gamma}}^1$-valued form,
                \item compose with $\omega_{\beta\gamma}$,
                \item then base change this composition to an $\Omega_{U_{\alpha\beta}}^1$-valued form;
            \end{enumerate}
            or do we instead
            \begin{enumerate}[(i)]
                \item base change both $\omega_{\gamma\delta}$ and $\omega_{\beta\gamma}$ to $\Omega_{U_{\alpha\beta}}^1$-valued forms,
                \item then perform the triple composition?
            \end{enumerate}
            That is:
            \[
                \omega_{\alpha\beta} \wedge M_{\alpha\beta}(\omega_{\beta\gamma} \wedge M_{\beta\gamma}\omega_{\gamma\delta}M_{\beta\gamma}^{-1})M_{\alpha\beta}^{-1} \overset{?}{=} \omega_{\alpha\beta} \wedge M_{\alpha\beta}\omega_{\beta\gamma}M_{\alpha\beta}^{-1} \wedge M_{\alpha\gamma}\omega_{\gamma\delta}M_{\alpha\gamma}^{-1}.
            \]

            The happy answer is that these two constructions are in fact equal, thanks to the cocycle condition on the $M_{\alpha\beta}$ and some form of associativity\footnote{That is, $A\wedge MB = AM\wedge B$, where $M$ is a matrix of $0$-forms.}, and so we can use whichever one we so please.
        \end{example}
    % subsection the_atiyah_class (end)

    \subsection{Simplicial forms} % (fold)
    \label{sub:simplicial_forms}
    
        \begin{note}
            For us, the $p$-simplex $\Delta_p$ is the ordered set $[p]=[0,1,\ldots,p]$, and its geometric realisation $|\Delta_p|$ is the \emph{smooth} space $\big\{(t_0,\ldots,t_p)\mid \sum_i t_i=1\big\}\subset\mathbb{R}^{n+1}$.
            We write $f_i\colon\Delta_{p-1}\to\Delta_p$ to mean the $i$-th face map.
        \end{note}

        \begin{definition}[Nerve of a cover]
            Given some topological space $Y$ with a cover $\mathscr{V}=\{V_\beta\}_{\beta\in J}$ we define the \emph{nerve $Y^\bullet_\mathscr{V}$} to be the simplicial space given by
            \[
                Y_\mathscr{V}^p = \bigsqcup_{\beta_0,\ldots,\beta_p\in J}\{V_{\beta_0\cdots\beta_p} \mid V_{\beta_0\cdots\beta_p}\neq\varnothing\}
            \]
            and with face maps acting by
            \[
                Y^p_\mathscr{V}f_i\colon V_{\beta_0\cdots\beta_p}\mapsto V_{\beta_0\cdots\hat{\beta}_i\cdots\beta_p}
            \]
            and degeneracy maps
            \[
                Y^p_\mathscr{V}g_i\colon V_{\beta_0\cdots\beta_p}\mapsto V_{\beta_0\cdots\beta_i\beta_i \cdots\beta_p}.\qedhere
            \]
        \end{definition}

        \begin{lemma}\label{lemma:forms-on-product-spaces}
            We have the \emph{local} decomposition
            \[
                \Omega_{Y\times Z}^r \cong \bigoplus_{i+j=r}\pi_Y^*\Omega_Y^i\otimes_{\O_{Y\times Z}}\pi_Z^*\Omega_Z^j
            \]
            where $\pi_Y$ and $\pi_Z$ are the projection maps.
        \end{lemma}

        \begin{proof}
            Taking small enough open sets $U\times V\subset X\times Y$, so that we have local coordinates over $U$ and over $V$, gives us the result almost immediately.
        \end{proof}

        \begin{definition}[Forms of type {$(i,j)$}]\label{definition:forms-of-type-ij}
            We write $\Omega^{i,j}_{Y\times Z} = \Omega_Y^i\otimes_{\O_{Y\times Z}}\Omega_Z^j$ and call its elements\footnote{Whenever there is no chance of confusion with the degrees coming from the Dolbeault bicomplex.} \emph{forms of type $(i,j)$}.
        \end{definition}

        \begin{definition}[Simplicial differential forms]
            A \emph{simplicial differential $r$-form} on a simplicial space $Y^\bullet$ is a collection $\omega = \big\{\omega_p\in\Omega_{|\Delta_p|\times Y^p}^r\big\}_{p\in\mathbb{N}}$ of forms that are {smooth} on $|\Delta_p|$ and {holomorphic} on $Y^p$ satisfying the condition
            \[
                (\id\times Y^\bullet f_i)^*\omega_{p-1} = (|f_i|\times\id)^*\omega_{p}\in\Omega_{|\Delta_{p-1}|\times Y^p}^r
            \]
            for all $p$.
            We write $\widetilde{\Omega}_{Y^\bullet}^r$ to mean the collection of all simplicial differential $r$-forms on $Y^\bullet$.
            Using \cref{definition:forms-of-type-ij} we refer to simplicial forms \emph{of type $(i,j)$ on $|\Delta_p\times Y^p|$} to mean a form of degree $i$ on the $p$-simplex and of degree $j$ on $Y^p$.
        \end{definition}

        The condition in the definition of simplicial differential forms is rather natural when compared to the fat realisation of of a simplicial space.
        This point of view is covered in \cite{Dupont:1976up}.

        \begin{lemma}[Fibre integration]\label{lemma:fibre-integration}
            There is a quasi-isomorphism
            \[
                \int_{|\Delta|}\colon \widetilde\Omega^r_{Y^\bullet} \congto \bigoplus_{p=0}^r\Omega^{r-p}_{Y^p}
            \]
            induced\footnote{See \cref{sec:the_dupont_isomorphism}.} by \emph{fibre integration}
            \[
                \int_{|\Delta_p|}\colon \Omega^r_{Y^\bullet} \congto \Omega^{r-p}_{Y^p}
            \]
            which is given by integrating the type $(p,r-p)$ part of a simplicial form over the geometric realisation of the $p$-simplex with its canonical orientation (see \cref{sec:the_dupont_isomorphism}).
        \end{lemma}

        \begin{proof}
            The classical proof is \cite[Theorem~2.3]{Dupont:1976up}, and the fact that the morphism is given by integrating over the simplices is mentioned in \cite[§2,~Remark~1]{Dupont:1976up}.
            However, this proof is for the smooth case.
            Although the proof for the holomorphic case is almost identical, we give it in \cref{sec:the_dupont_isomorphism} (for the case $Y^\bullet=X^\bullet_\cover$) so as to not miss the myriad of confusions that can arise from choices of orientation and signs.
        \end{proof}

        \begin{example}[Simplicial differential forms on the nerve]\label{example:fibre-integration-simplicial-forms-nerve}
            Taking $Y^\bullet=X^\bullet_\cover$ gives
            \[
                \int_{|\Delta|}\colon \widetilde\Omega^r_{X^\bullet_\cover} \congto \bigoplus_{p=0}^r\Omega^{r-p}_{X^p_\cover} \cong \Tot^r\cech^\bullet(\Omega^\bullet_X).
            \]
            It is interesting to note that the conditions we impose on $\cover$ are only to ensure that this quasi-isomorphism \emph{calculates} de-Rham cohomology; the actual quasi-isomorphism in \cref{lemma:fibre-integration} does \emph{not} depend on the choice of cover.
        \end{example}
    
    % subsection simplicial_forms (end)

% section preliminaries (end)



\section{Manual construction} % (fold)
\label{sec:manual_construction}

    It is a classical fact\footnote{Since $\cover$ is Stein (and any finite intersection of Stein open sets is also Stein) and $\E$ is coherent, Cartan's Theorem B tell us that $\check{\mathbb{H}}^k(\cover,\Omega_X^\bullet)\cong\check{\mathbb{H}}^k(X,\Omega_X^\bullet)$; since $X$ is paracompact we know that $\check{\mathbb{H}}^k(X,\Omega_X^\bullet)\cong\mathbb{H}^k(X,\Omega_X^\bullet)$; and \cite[Theorem~8.1]{Voisin:2002wn} tells us that $\mathbb{H}^k(X,\Omega_X^\bullet)\cong\H^k(X,\mathbb{C})$.} that $H^r\Tot^\bullet\cech^\circ(\Omega^\circ_X) \cong \H^r(X,\mathbb{C})$.
    Say (as will be the case with the Atiyah class) we are given some $\mathfrak{a}_0\in\cech^p(\Omega_X^p)$, with $\check{\delta}\mathfrak{a}_0=0$ but $\d\mathfrak{a}_0\neq0$.
    Define $\mathfrak{a}_p=0\in\cech^0(\Omega_X^{2p})$.
    If we can find $\mathfrak{a}_i\in\cech^{p-i}(\Omega^{p+i}_X)$ for $i=1,\ldots,(p-1)$ such that $\check\delta\mathfrak{a}_i=\d\mathfrak{a}_{i-1}$ then
    \[
        (0,\pm\mathfrak{a}_{p-1},\ldots,\pm\mathfrak{a}_1,\mathfrak{a}_0,0,\ldots,0) \in \Tot^{2p}\cech^\bullet(\Omega^\bullet_X)
    \]
    is $(\check\delta\pm\d)$-closed\footnote{The signs depend on the Čech degree: the total differential, acting on an $\ell$-cocycle of $k$-forms, is $(-1)^\ell\d+\check\delta$.}, and thus represents a cohomology class in $\H^{2p}\Tot^\bullet\cech^\circ(\Omega_X^\circ)$, and thus a cohomology class in $\H^{2p}(X,\mathbb{C})$.

    \begin{note}
        Although we have the isomorphism $H^r\Tot^\bullet\cech^\circ(\Omega^\circ_X) \cong \H^r(X,\mathbb{C})$, we don't necessarily have an easy way of computing explicitly what a closed class in the Čech-de Rham complex maps to, \emph{unless} it has a non-zero degree-$(0,r)$ part, in which case it maps exactly to this.
        There is an important difference in our approach to closed de-Rham classes and Deligne classes: what we construct in \cref{prt:vector_bundles_deligne} is `clearly' the Chern class, in particular because it has a non-zero degree-$(0,r)$\footnote{There is a slight change in the degree-labelling of Deligne cohomology, but this is just a technicality.} part that `is' the Chern class; but what we construct here has no such term, and so we need to show that this lift of the Atiyah class (which `is' the Chern class) really \emph{still is} the Chern class, and this is the purpose of \cref{sec:axiomatic_chern_classes}.
    \end{note}
    
    \subsection{The first Atiyah class} % fold
    \label{sub:the_first_atiyah_class}

        We wish to calculate $(\nabla_\beta-\nabla_\alpha)$ in the basis $\{s^\alpha_1,\ldots,s^\alpha_\mathfrak{r}\}$ and express it as some endomorphism-valued $1$-form $\omega_{\alpha\beta}$, so that $\at_\E=[\{\omega_{\alpha\beta}\}_{\alpha,\beta}]$.
        \begin{align*}
            (\nabla_\beta-\nabla_\alpha)(s^\alpha_k) = \nabla_\beta s^\alpha_k
            &= \nabla_\beta\sum\nolimits_\ell(M_{\alpha\beta})_k^\ell s^\beta_\ell\\
            &= \sum\nolimits_\ell \big( \nabla_\beta (s^\beta_\ell) (M_{\alpha\beta})_k^\ell + s^\beta_\ell\otimes\mathrm{d}(M_{\alpha\beta})_k^\ell \big)\\
            &= \sum\nolimits_\ell \big( s^\beta_\ell\otimes \mathrm{d}(M_{\alpha\beta})_k^\ell \big)\\
            &= \sum\nolimits_\ell \Big( \left(\sum\nolimits_m (M_{\alpha\beta}^{-1})_\ell^m s^\alpha_m\right)\otimes \mathrm{d}(M_{\alpha\beta})_k^\ell \Big)\\
            &= \sum\nolimits_{\ell,m} s^\alpha_m\otimes (M_{\alpha\beta}^{-1})_\ell^m\mathrm{d}(M_{\alpha\beta})_k^\ell\\
            &= \sum\nolimits_m \big( s^\alpha_m\otimes M_{\alpha\beta}^{-1}\mathrm{d}(M_{\alpha\beta}) \big)_k^m.
        \end{align*}
        
        This means that $(\nabla_\beta-\nabla_\alpha)$ is given by the matrix of $1$-forms
        \begin{equation}\label{equation:nabla-minus-nabla-omega}
            \omega_{\alpha\beta} = M_{\alpha\beta}^{-1}\mathrm{d}M_{\alpha\beta}.
        \end{equation}
    
        \begin{lemma}
            The element $\tr(\at_\E)$ is $\mathrm{d}$-closed.
            More precisely, $\{\mathrm{d}\tr(\omega_{\alpha\beta})\}_{\alpha,\beta}\in\cech^1(\Omega^2_X)$ is zero.
        \end{lemma}
        
        \begin{proof}
            Note the following two facts: if $\varphi(m)=m^{-1}$ then $d\varphi(m)(h)=-m^{-1}hm^{-1}$; and $\tr$ and $\mathrm{d}$ commute.
            Thus
            \begin{align*}
                \mathrm{d}\tr(\omega_{\alpha\beta}) &= \tr(\mathrm{d}\omega_{\alpha\beta})\\
                &= \tr(\d M_{\alpha\beta}^{-1}\d M_{\alpha\beta} - M_{\alpha\beta}^{-1}\d^2M_{\alpha\beta})\\
                &= \tr(\d M_{\alpha\beta}^{-1}\d M_{\alpha\beta})\\
                &= \tr(-M_{\alpha\beta}^{-1}\d M_{\alpha\beta}M_{\alpha\beta}^{-1}\d M_{\alpha\beta})\\
                &= -\tr(\omega_{\alpha\beta}^2).
            \end{align*}
            But now, since $\omega_{\alpha\beta}$ is a matrix of $1$-forms, which are skew-symmetric, we see that
            \[
                -\tr(\omega_{\alpha\beta}^2) = -\sum\nolimits_{a,b}(\omega_{\alpha\beta})^a_b \wedge (\omega_{\alpha\beta})^b_a = 0.\qedhere
            \]
        \end{proof}

        \begin{lemma}
            The element $\tr(\at_\E)$ is $\check\delta$-closed.
            More precisely, $\{\check\delta\tr(\omega_{\alpha\beta})\}_{\alpha,\beta}\in\cech^2(\Omega^1_X)$ is zero.
        \end{lemma}

        \begin{proof}
            \[
                \check\delta\tr(\omega_{\alpha\beta}) = \tr(\omega_{\beta\gamma}-\omega_{\alpha\gamma}+\omega_{\alpha\beta})=0
            \]
            where the second equality is shown in \cref{sub:the_second_atiyah_class}.
        \end{proof}
        
        This calculation can be summarised in the following diagram:
        \begin{equation}
            \begin{tikzcd}[row sep=large]
                0
                &\\
                \tr(\omega_{\alpha\beta})
                    \ar{u}{\mathrm{d}}
                    \ar{r}{\check\delta}
                & 0
            \end{tikzcd}
        \end{equation}

    % subsection the_first_atiyah_class (end)
    


    \subsection{The second Atiyah class} % fold
    \label{sub:the_second_atiyah_class}
        
            By \cref{example:second-atiyah-class} we know that $\at_\E^2 = [\{\omega_{\alpha\beta}M_{\alpha\beta}\omega_{\beta\gamma}M_{\alpha\beta}^{-1}\}_{i,j}]$.
            We introduce the following notation: $A=\omega_{\alpha\beta}$, $B=\omega_{\alpha\gamma}$, $M=M_{\alpha\beta}$, $X=M\omega_{\beta\gamma}M^{-1}$.
            Thus $\at_\E^2=AX$.
            Now, $\omega_{\alpha\beta} = M_{\alpha\beta}^{-1}\d M_{\alpha\beta}$, whence $\mathrm{d}A = -A^2$, and similarly for $B$ and $X$.
            Further, by differentiating the cocycle condition $M_{\alpha\beta}M_{\beta\gamma} = M_{\alpha\gamma}$ and right-multiplying by $M_{\alpha\gamma}^{-1}$, we see that\footnote{As we already said, $M\omega_{\beta\gamma}M^{-1}$ is the natural way of thinking of $X$ as being a map \textit{into} something lying over $U_i$, so this equation should be read as a cocycle condition over $U_i$ by thinking of it as $\omega_{\alpha\beta}+\tilde{\omega}_{\beta\gamma}=\omega_{\alpha\gamma}$, where the tilde corresponds to a base change (see \cref{note:relations-with-circeq} for a `better' statement). Note that this is also the result we expect, since $\omega_{\alpha\beta}$ corresponds to $\nabla_\beta-\nabla_\alpha$, and this clearly satisfies the additive cocycle condition, as already shown.} \mbox{$A+X=B$}.
            Hence
            \[
                \at_\E^2=A(B-A).
            \]
            Using the fact that $\mathrm{d}A=A^2$ we see that $\mathrm{d}A^2=0$, whence
            \[
                \mathrm{d}\tr(\at_\E^2) = \mathrm{d}\tr(AB-A^2) = -\tr(A^2B-AB^2).
            \]

            We want $f\in\cech^1(\Omega_X^3)$ such that $\delta f=-\mathrm{d}\tr(\at_\E^2)$ and $\d f=0$.
            It is clear that we need, at least, $f$ to be (the trace of) a polynomial of homogeneous degree $3$ in the one variable $A=\omega_{\alpha\beta}$.
            But then $f(A)=\tr(A^3)$ is, up to a scalar multiple, our only option.
            We set $f(A)=\frac13\tr(A^3)$ and compute its Čech coboundary:
            \begin{align*}
                (\delta f)_{\alpha\beta\gamma} &= f(\omega_{\beta\gamma})-f(\omega_{\alpha\gamma})+f(\omega_{\alpha\beta})\\
                &= f(B-A)-f(B)+f(A)\\
                &= \frac13\tr\big((B-A)^3-B^3+A^3\big)\\
                &= \frac13\tr(-AB^2 -BAB - B^2A + A^2B + ABA + BA^2)
            \end{align*}

            \begin{lemma}\label{lemma:cyclic-permutation-under-trace}
                Let $x_1,\ldots,x_n$ be square matrices of $1$-forms.
                Then \[
                    \tr(x_1x_2\cdots x_n) = (-1)^{n-1}\tr(x_2x_3\cdots x_nx_1).\qedhere
                \]
            \end{lemma}
            
            \begin{proof}
                This is just using the anti-commutativity of the wedge product:
                \begin{align*}
                    \tr(x_1x_2\cdots x_n) &= \sum_{a_i}(x_1)^{a_1}_{a_2}\wedge(x_2)^{a_2}_{a_3}\wedge\cdots\wedge(x_n)^{a_n}_{a_1}\\
                    &= -\sum_{a_i}(x_2)^{a_2}_{a_3}\wedge(x_1)^{a_1}_{a_2}\wedge\cdots\wedge(x_n)^{a_n}_{a_1}\\
                    = \ldots &= (-1)^{(n-1)} \sum_{a_i}(x_2)^{a_2}_{a_3}\wedge(x_3)^{a_3}_{a_4}\wedge\cdots\wedge(x_n)^{a_n}_{a_1}\wedge(x_1)^{a_1}_{a_2}\\
                    &= (-1)^{n-1}\tr(x_2x_3\cdots x_nx_1).\qedhere
                \end{align*}
            \end{proof}
            
            Using \cref{lemma:cyclic-permutation-under-trace} we see that
            \begin{align*}
                (\delta f)_{\alpha\beta\gamma} &= \frac13\tr(-AB^2-AB^2-AB^2+A^2B+A^2B+A^2B)\\
                &= \tr(A^2B - AB^2)\\
                &= -\mathrm{d}\tr(\at_\E^2).
            \end{align*}
            Now we just have to worry about whether or not $\mathrm{d}f$ is zero.
            But
            \begin{align*}
                \mathrm{d}f(A) &= \mathrm{d}\frac13\tr(A^3)\\
                &= \frac13\tr(\mathrm{d}A A^2-A\mathrm{d}A^2)\\
                &= -\frac13\tr(A^4),
            \end{align*}
            and we know\footnote{By exactly the same argument for showing that $\tr(A^2)=0$ where $A$ is a matrix of $1$-forms: we can cyclically permute to obtain that $\tr(A^{2n})=\tr(A^{2n-1}A)=-\tr(AA^{2n-1})$, whence $\tr(A^{2n})=0.$} that $\tr(A^{2n})=0$ for any integer $n$, so we are done.

            Finally, note that\footnote{But really this is superfluous: $\omega_{\alpha\beta}$ is a $1$-cocycle by definition.}
            \begin{align*}
                \check\delta\tr\big(\omega_{\alpha\beta}(\omega_{\alpha\gamma}-\omega_{\alpha\beta})\big)
                &= \check\delta\tr(\omega_{\alpha\beta}\omega_{\alpha\gamma})\\
                &= \tr(\omega_{\beta\gamma}\omega_{\beta\delta}) - \tr(\omega_{\alpha\gamma}\omega_{\alpha\delta}) + \tr(\omega_{\alpha\beta}\omega_{\alpha\delta}) - \tr(\omega_{\alpha\beta}\omega_{\alpha\gamma})\\
                &= \tr\big((\omega_{\alpha\gamma}-\omega_{\alpha\beta})(\omega_{\alpha\delta}-\omega_{\alpha\beta})\big) - \tr(\omega_{\alpha\gamma}\omega_{\alpha\delta}) + \tr(\omega_{\alpha\beta}\omega_{\alpha\delta}) - \tr(\omega_{\alpha\beta}\omega_{\alpha\gamma})\\
                &= -\tr(\omega_{\alpha\gamma}\omega_{\alpha\beta}) + \tr(\omega_{\alpha\beta}^2) - \tr(\omega_{\alpha\beta}\omega_{\alpha\gamma})\\
                &= \tr(\omega_{\alpha\beta}\omega_{\alpha\gamma}) - \tr(\omega_{\alpha\beta}\omega_{\alpha\gamma}) = 0.
            \end{align*}

            This calculation can be summarised by the following commutative diagram:
            \begin{equation}\label{equation:second-atiyah-class-summary}
                \begin{tikzcd}[row sep=large]
                    0&&\\
                    -\frac13\tr(A^3)
                        \ar{u}{\mathrm{d}} 
                        \ar{r}{\check\delta}
                    &\tr(A(B-A)B)
                    &\\
                    &\underbrace{\tr\big(A(B-A)\big)}_{\tr\at_\E^2}
                        \ar{u}{\mathrm{d}}
                        \ar{r}{\check\delta}
                    & 0
                \end{tikzcd}
            \end{equation}
            Tahis gives us the closed element
            \begin{equation}
                \tr\at_\E^2 \mapsto \left(0, \frac13\tr(A^3), \tr\big(A(B-A)\big),0,0\right)\in\Tot^4\cech^\bullet(\Omega_X^\bullet)
            \end{equation}

    % subsection the_second_atiyah_class (end)
    


    \subsection{The third Atiyah class} % fold
    \label{sub:the_third_atiyah_class}

            We now extend the notation from \cref{sub:the_second_atiyah_class}: write
            \begin{align*}
                &A=\omega_{\alpha\beta} \qquad M=M_{\alpha\beta} \qquad X=M\omega_{\beta\gamma}M^{-1}\\
                &B=\omega_{\alpha\gamma} \qquad N=M_{\alpha\gamma} \qquad Y=N\omega_{\gamma\delta}N^{-1}\\
                &C=\omega_{\alpha\delta}
            \end{align*}
            so that $\at_\E^3 = AXY = A(B-A)(C-B)$.

            The first step is simple:
            \[
                \d\tr\at_\E^3 = -\tr\big(A(B-A)(C-B)C\big)\in\cech^3(\Omega^4_X).
            \]
            Now, trying to find some $\varphi\in\cech^2(\Omega^4_X)$ such that $\check\delta\varphi=\d\tr\at_\E^3$ is slightly harder.
            The most naïve approach is to find all the monomials in $\cech^2(\Omega^4_X)$, apply the Čech differential, and equate coefficients.
            Using the fact that we can cyclically permute under the trace, finding all the monomials is the same as finding all degree $2$ monomials in non-commutative variables $X$ and $Y$, modulo equivalence under cyclic permutation, and there are just four of these: $X^2Y^2$, $(XY)^2$, $X^3Y$, and $XY^3$.
            Thus, we find that
            \begin{equation}\label{equation:2-4-term-manual-construction}
                \varphi = - \frac14\tr\Big(\big(A(B-A)\big)^2\Big) + \frac12 \tr(A^2(B-A)^2) - \frac12\tr\big(A^3(B-A)\big) - \frac12\tr\big(A(B-A)^3\big).
            \end{equation}
            Noting that $\d\varphi$ can be factored as $\frac{1}{10}\tr\big((B-A)^5-B^5+A^5\big)$ we see that
            \begin{equation}\label{equation:1-5-term-manual-construction}
                \d\varphi = \check\delta\frac{1}{10}\tr(A^5).
            \end{equation}

            This calculation can be summarised in the following commutative diagram:
            \begin{equation}\label{equation:third-atiyah-class-summary}
                \begin{tikzcd}[row sep=huge,column sep=small]
                    0\\
                    \frac{1}{10}\tr(A^5)
                        \ar{u}{\mathrm{d}} 
                        \ar{r}{\check\delta}
                    & 
                    \frac{1}{10}\tr\big((B-A)^5 - B^5 + A^5\big)\\
                    & -\frac14\tr\Big(\big(A(B-A)\big)^2\Big) + \frac12 \tr(A^2(B-A)^2) 
                        \ar{u}{\mathrm{d}}
                    &\\[-4em]
                    & - \frac12\tr\big(A^3(B-A)\big) - \frac12\tr\big(A(B-A)^3\big)
                        \ar{r}{\check\delta}
                    & -\tr\big(A(B-A)(C-B)C\big) \\
                    && \underbrace{\tr\big(A(B-A)(C-B)\big)}_{\at_\E^3}
                        \ar{u}{\mathrm{d}}
                        \ar{r}{\check\delta}
                    & 0
                \end{tikzcd}
            \end{equation}
            Taking the signs of the total differential into account, this gives us the closed element
            \begin{equation}
                \tr\at_\E^3 \mapsto \left(
                    0,-\frac{1}{10}\tr(A^5),
                    \rho(A,X),
                    \tr\big(AXY\big),0,0,0
                \right)\in\Tot^6\cech^\bullet(\Omega_X^\bullet)
            \end{equation}
            where
            \begin{equation*}
                \rho(A,X) = - \frac14\tr\big(AXAX\big) + \frac12 \tr\big(A^2X^2\big) - \frac12\tr\big(A^3X\big) - \frac12\tr\big(AX^3\big).
            \end{equation*}
            
            \begin{definition}
                To avoid having to write $\tr$ everywhere, we use the notation $\circeq$ to mean `equal up to a cyclic permutation and corresponding sign', i.e.
                \[
                    x_1x_2\cdots x_n \circeq (-1)^{n+1}x_2x_3\cdots x_nx_1 \circeq (-1)^{m(n+1)}x_mx_{m+1}\cdots x_nx_1x_2\cdots x_{m-1}.\qedhere
                \]
            \end{definition}

            \begin{note}\label{note:relations-with-circeq}
                Using the above notation we have the following relations:
                \begin{enumerate}[(i)]
                    \item $\omega_{\alpha\beta}^2\circeq0$;
                    \item $M\omega_{\alpha\beta}M^{-1}\circeq\omega_{\alpha\beta}$ for all invertible matrices $M$.
                \end{enumerate}
                Thus $\omega_{\alpha\beta}+\omega_{\beta\gamma}\circeq\omega_{\alpha\gamma}$.
            \end{note}

    % subsection the_third_atiyah_class (end)

    \subsection{The fourth Atiyah class} % (fold)
    \label{sub:the_fourth_atiyah_class}

        There are clearly patterns that we can spot looking at \cref{equation:second-atiyah-class-summary,equation:third-atiyah-class-summary}, such as the fact that the Čech $1$-cocycle seems to be (some constant multiple of) $\tr(A^{2k-1})$.
        But beyond this, trying to work in full generality with the $k$-th Atiyah class is difficuly, no small part thanks to the large number of terms.
        It seems believable that the even Atiyah classes and the odd Atiyah classes would follow different patterns, but unfortunately we don't have many explicit cases to study: $k=0,1$ are both trivial, and $k\geqslant4$ is too large for us be able to spot any patterns.
        For completeness we give the lift of the fourth Atiyah class below, which was calculated using a Haskell program written by the author\footnote{It just uses the naïve approach from \cref{sub:the_third_atiyah_class} of repeatedly calculating all of the non-commutative monomials, modulo cyclic permutations, applying the Čech derivative, and then equating coefficients.}, but after this we will look for a different method of finding a lift, using simplicial data.

        \bigskip

        The element in the total complex is
        \begin{equation}
            \Big(0,-\frac{1}{35}\tr\at_\E^{4,(1,7)},\frac15\tr\at_\E^{4,(2,6)},\frac15\tr\at_\E^{4,(3,5)},\tr\at_\E^{4,(4,4)},0,0,0,0\Big)
        \end{equation}
        where
        \begin{align*}
            \at_\E^{4,(4,4)}
            &= A(B-A)(C-B)(D-C)\\
            \at_\E^{4,(3,5)}
            &=
            \frac{13}{5} A^5
            +13 A^4(B-A)
            +5 A^3(B-A)^2
            +5 A^3(B-A)(C-A)\\
            &\quad+3 A^3(C-A)(B-A)
            +4 A^2(B-A)A(B-A)
            +4 A^2(B-A)A(C-A)\\
            &\quad+3 A^2(B-A)^3
            - A^2(B-A)^2(C-A)
            +5A^2(B-A)(C-A)^2\\
            &\quad+5A^2(C-A)A(B-A)
            +2 A^2(C-A)(B-A)^2
            + A^2(C-A)(B-A)(C-A)\\
            &\quad+3 A^2(C-A)^2(B-A)
            - A(B-A)A(C-A)(B-A)
            +5 A(B-A)A(C-A)^2\\
            &\quad-5 A(B-A)^2(C-A)(B-A)
            +5 A(B-A)(C-A)A(C-A)
            +5 A(B-A)(C-A)^3\\
            &\quad+4 \big(A(C-A)\big)^2(B-A)
            -2 A(C-A)(B-A)^3
            +4 A(C-A)(B-A)^2(C-A)\\
            &\quad+ A\big((C-A)(B-A)\big)^2
            +2 A(C-A)^2(B-A)^2
            + A(C-A)^2(B-A)(C-A)\\
            &\quad+3 A(C-A)^3(B-A)\\
            \at_\E^{4,(2,6)}
            &=
            5 A^5(B-A)
            -4 A^4(B-A)^2
            + A^3(B-A)A(B-A)
            + A^3(B-A)^3\\
            &\quad-5 A^2(B-A)A(B-A)^2
            -4 A^2(B-A)^2A(B-A)
            -4 A^2(B-A)^4\\
            &\quad+\frac{1}{3} \big(A(B-A)\big)^3
            + A(B-A)A(B-A)^3
            + A(B-A)^5\\
            \at_\E^{4,(1,7)}
            &= A^7
        \end{align*}

    % subsection the_fourth_atiyah_class (end)

% section manual_construction (end)



\section{Simplicial construction} % (fold)
\label{sec:simplicial_construction}

    \subsection{The global simplicial connection} % (fold)
    \label{sub:the_global_simplicial_connection}
            
        Write $\pi\colon |\Delta_p|\times X^p_\cover\to X^p_\cover$ to mean the projection map; write $\E^p$ to mean $\E$ as a sheaf on $X^p_\cover$; and write $Y^p=|\Delta_p|\times X^p_\cover$.
        Define $\overline{\E^p} = \pi^*\E^p = \E^p\otimes_{\O_{X^p_\cover}}\O_{Y^p}$.

        \begin{note}\label{note:how-does-ti-act-on-e}
            The map $\O_{|\Delta_p|}\to\O_{Y^p}$ gives an $\O_{|\Delta_p|}$-action on $\E^p$.
            Further, we can use $\pi$ to pull back sections, and so extend the $\nabla_\alpha$ to connections on $\overline{\E^p}$.           
        \end{note}

        % \begin{definition}[Simplicial connection]
        %     Let $Z^\bullet$ be a simplicial space and $\mathcal{G}_\bullet$ a simplicial vector bundle\footnote{Simply a vector bundle $\mathcal{G}_p$ on each $Z^p$ along with a morphism $\mathcal{G}_\bullet(f)\colon\mathcal{G}_m\to\mathcal{G}_n$ for each $f\colon\Delta_m\to\Delta_n$, also satisfying $\mathcal{G}_\bullet(g\circ f)=\mathcal{G}_\bullet(g)\circ\mathcal{G}_\bullet(f)$.} on $Z^\bullet$.
        %     A \emph{simplicial connection on $\mathcal{G}_\bullet$} is a collection $\{\nabla^{(p)}\}_{p\in\mathbb{N}}$ of connections on each $\mathcal{G}_p$ such that
        %     \[
        %         \big(\mathcal{G}_\bullet(f_i)\otimes\id\big)\circ\nabla^{(p-1)} = \nabla^{(p)}\circ\mathcal{G}_\bullet(f_i)\colon \mathcal{G}_{p-1}\to\mathcal{G}_{p}\otimes\Omega^1_{\mathcal{G}_{p}} {\color{red}\textbf{!!!???}}
        %     \]
        %     for all face maps $f_i\colon\Delta_{p-1}\to\Delta_{p}$.
        % \end{definition}

        \begin{definition}[Global simplicial connection]\label{definition:global-simplicial-connection}
            Using \cref{note:how-does-ti-act-on-e} we can define
            \[
                \simpconn^{(p)} = \sum_{i=0}^p t_i\nabla_{\alpha_i}\colon \overline{\E^p}\to\overline{\E^p}\otimes\Omega^1_{|\Delta_p|\times X^p_\cover}
            \]
            which acts on a section $s\otimes\varphi$ of $\overline{\E^p}$ over $U_{\alpha_0\ldots\alpha_p}$ by
            \begin{align*}
                \simpconn^{(p)} (s\otimes\varphi) = \sum_{i=0}^p t_i\nabla_{\alpha_i}(s\otimes\varphi)
                &= \sum_i t_i\nabla_{\alpha_i}(\varphi s\otimes 1)\\
                &= \sum_i t_i\big(\varphi\pi^*\big(\nabla_{\alpha_i}(s)\big) + s\otimes 1\otimes\d\varphi\big)\\
                &= \sum_i \pi^*\big(t_i\varphi\otimes\nabla_{\alpha_i}(s)\big) + s\otimes 1\otimes t_i\d\varphi.\qedhere
            \end{align*}
        \end{definition}

        % \begin{note}
        %     In some sense that we don't make precise, $\simpconn=\{\simpconn^{(p)}\}_{p\in\mathbb{N}}$ is a \emph{simplicial connection} \mbox{on $X^\bullet_\cover$.}
        % \end{note}

        Note that we have an alternative expression for $\simpconn^{(p)}$ given by
        \begin{equation}\label{equation:alternative-expression-for-simpconn}
            \simpconn^{(p)} = \nabla_{\alpha_0} + \sum_{i=1}^p t_i(\nabla_{\alpha_i}-\nabla_{\alpha_0}) = \nabla_{\alpha_0} + \sum_{i=1}^p t_i\omega_{\alpha_0\alpha_i}.
        \end{equation}
        So, recalling \cref{equation:nabla-minus-nabla-omega}, and that the $s_{\alpha_i}^\ell$ are $\nabla_{\alpha_i}$-flat, we see that the $p$-th simplicial-level curvature acts by
        \begin{align*}
            \kappa\big(\simpconn^{(p)}\big)(s_{\alpha_0}^\ell) &= \Big(\nabla_{\alpha_0} + \sum_{i=1}^p t_i\omega_{\alpha_0\alpha_i}\Big)^2(s_{\alpha_0}^\ell)\\
            &= \nabla_{\alpha_0}^2(s_{\alpha_0}^\ell) + \sum_{i=1}^p \Big[\big(\nabla_{\alpha_0}(s_{\alpha_0}^\ell)\wedge t_i\omega_{\alpha_0\alpha_i}\big)\otimes + s_{\alpha_0}^\ell\otimes\d(t_i\omega_{\alpha_0\alpha_i})\Big]\\
            &\quad\,+ \sum_{i=1}^p\big(t_i\omega_{\alpha_0\alpha_i}\wedge\nabla_{\alpha_0}(s_{\alpha_0}^\ell)\big) + \sum_{i,j=1}^p s_{\alpha_0}^\ell\otimes(t_jt_i\omega_{\alpha_0\alpha_j}\omega_{\alpha_0\alpha_i})\\
            &= \sum_{i=1}^p s_{\alpha_0}^\ell\otimes\d(t_i\omega_{\alpha_0\alpha_i}) + \sum_{i,j=1}^p s_{\alpha_0}^\ell\otimes(t_jt_i\omega_{\alpha_0\alpha_j}\omega_{\alpha_0\alpha_i})
        \end{align*}
        which is simply\footnote{Recall (\cref{sec:the_dupont_isomorphism}) that the de Rham differential on a form of type $(i,j)$ is given by $\d_{t}+(-1)^i\d_X$, where $\d_t$ is the differential in the $t_i$. Hence, since $\overline{\omega}_{(p)}$ is a (matrix of) $(0,1)$ forms, the differential is simply `the usual one', i.e. we use the product rule (which is the same as applying $\d_t+\d_X$)} $\d\overline\omega_{(p)}+\overline{\omega}^2_{(p)}$ on $U_{\alpha_0\cdots\alpha_p}$ in the $U_{\alpha_0}$ trivialisation, where $\overline{\omega}_{(p)}$ is the matrix of $(0,1)$-forms given by $\sum_{i=1}^p t_i\omega_{\alpha_0\alpha_i}$.
        Then
        \[
            \d\overline{\omega}_{(p)} + \overline{\omega}_{(p)}^2 \in \Gamma\big(X^p_\cover,\Omega^2_{|\Delta_p|\times X^p_\cover}(\sheafend({\overline{\E^p}}))\big).
        \]

        \begin{note}
            Note that $\sheafend(\overline{\E^p})$-valued forms on $|\Delta_p|\times X^p_\cover$ are (locally) `usual' forms on $|\Delta_p|$ times the identity matrix\footnote{That is, we write $\d t_i$ when really we mean $\d t\cdot I_\mathfrak{r}$.}, tensored with $(\mathfrak{r}\times \mathfrak{r})$-matrix-valued forms on $X^p_\cover$.
        \end{note}

        \begin{definition}[Simplicial Atiyah class]\label{definition:simplicial-atiyah-class}
            We define the \emph{$k$-th simplicial Atiyah class} as
            \[
                \underline{\at}^k_\E = \bigg\{\epsilon_k\big(\d\overline{\omega}_{(p)} + \overline{\omega}_{(p)}^2\big)^k \in \Gamma\Big(X^p_\cover,\,\Omega_{|\Delta_p|\times X^p_\cover}^{2k}\big(\sheafend(\overline{\E^p})\big)\Big)\bigg\}_{p\in\mathbb{N}}
            \]
            where $\overline{\omega}_{(p)} = \sum_{i=1}^p t_i\omega_{\alpha_0\alpha_i}$ and $\epsilon_k=(-1)^{k(k-1)/2}$.
        \end{definition}

        \begin{note}\label{note:sign-in-simplicial-atiyah-class}
            The sign $\epsilon_k$ in \cref{definition:simplicial-atiyah-class} comes from the fact that we want the wedge product of simplicial forms to respect fibre integration in some sense.
            Here, we want the sign of the $(k,k)$-term of $\tr\int_{|\Delta|}\underline{at}^k_\E$ to agree with the sign of the Čech $k$-cocycle of $k$-forms $\at^k_\E$.
            As shown in the proof of \cref{theorem:k-k-term-of-simplicial-atiyah-class}, the $(k,k)$-term involves changing the sign $T_{k-1}$ times, where $T_{k-1}$ is the $(k-1)$-th triangle number.
        \end{note}

        \begin{theorem}
            The $k$-th simplicial Atiyah class defines a simplicial form $\underline{\at}_\E^k(U_\alpha)$ over every $U_{\alpha}\in\cover$.
            That is, we can write
            \[
                \underline{\at}_\E^k \in \Gamma\left(X^\bullet_\cover,\,\widetilde{\Omega}^{2k}_{X^\bullet_\cover}\big(\sheafend(\overline{\E^\bullet})\big)\right).\qedhere
            \]
        \end{theorem}

        \begin{proof}
            Since $\d$ and the wedge product both commute with pullbacks, we see that both $\d\mu$ and $\mu\wedge\mu$ are simplicial forms if $\mu$ is (and clearly $-\mu$ is also a simplicial form).
            Thus it only remains to show that $\overline{\omega}_{(p)}=\sum_{i=1}^p t_i\omega_{\alpha_0\alpha_i}$ is a simplicial form, by showing that, for all face maps $f_j\colon\Delta_{p-1}\to\Delta_p$, we have the equality
            \[
                (\id\times|f_j|)^*\overline{\omega}_{(p)} = (X^\bullet_\cover(f_j)\times\id)^*\overline{\omega}_{(p-1)} \in \Omega^1_{|\Delta_{p-1}|\times X^p_\cover}\big(\sheafend(\overline{\E^p})\big).
            \]
            But this is just asking that $\overline{\omega}_{(p)} = \overline{\omega}_{(p-1)}$ on the face $\{t_j=0\}$ of $|\Delta_p|\times U_{\alpha_0\cdots\alpha_p}$ for any $U_{\alpha_0\cdots\alpha_p}\in X^p_\cover$, and there both sides are equal to $\sum_{i=1,i\neq j}^p t_i\omega_{\alpha_0\alpha_i}$.
        \end{proof}

        \begin{note}
            There is no reason to expect the simplicial Atiyah class to be a cocycle \emph{before} applying fibre integration.
        \end{note}

        \begin{lemma}
            The trace of the $k$-th simplicial Atiyah class is $\d$-closed.
        \end{lemma}

        \begin{proof}
            It suffices (thanks to the product rule) to show that the trace of the first simplicial Atiayh class is $\d$-closed, and
            \begin{align*}
                \d\big(\d\overline{\omega}_{(p)}+\overline{\omega}_{(p)}^2\big) &= \d^2\overline{\omega}_{(p)} + \d\overline{\omega}_{(p)}^2\\
                &= \d\Big(\sum_{i,j=1}^p t_it_j\omega_{\alpha_0\alpha_i}\omega_{\alpha_0\alpha_j}\Big)\\
                &= \sum_{i,j=1}^p \Big[\d_{|\Delta|}(t_it_j)\otimes\omega_{\alpha_0\alpha_i}\omega_{\alpha_0\alpha_j} + t_it_j\d_X(\omega_{\alpha_0\alpha_i}\omega_{\alpha_0\alpha_j})\Big]\\
                &= \sum_{i,j=1}^p \Big[t_j\d_{|\Delta|}t_i\otimes\omega_{\alpha_0\alpha_i}\omega_{\alpha_0\alpha_j} + t_i\d_{|\Delta|}t_j\otimes\omega_{\alpha_0\alpha_i}\omega_{\alpha_0\alpha_j}\\
                &\qquad\qquad + t_it_j\d_X(\omega_{\alpha_0\alpha_i})\omega_{\alpha_0\alpha_j} - t_it_j\omega_{\alpha_0\alpha_i}\d_X(\omega_{\alpha_0\alpha_j})\Big]\\
                &\circeq \sum_{i,j=1}^p \Big[t_j\d_{|\Delta|}t_i\otimes\omega_{\alpha_0\alpha_i}\omega_{\alpha_0\alpha_j} - t_i\d_{|\Delta|}t_j\otimes\omega_{\alpha_0\alpha_j}\omega_{\alpha_0\alpha_i}\\
                &\qquad\qquad - t_it_j\omega_{\alpha_0\alpha_i}^2\omega_{\alpha_0\alpha_j} + t_it_j\omega_{\alpha_0\alpha_i}\omega_{\alpha_0\alpha_j}^2\Big]\\
                &\circeq \sum_{i,j=1}^p \Big[-t_it_j\omega_{\alpha_0\alpha_j}\omega_{\alpha_0\alpha_i}^2 + t_it_j\omega_{\alpha_0\alpha_i}\omega_{\alpha_0\alpha_j}^2\Big] =0.\qedhere
            \end{align*}
        \end{proof}
    
    % subsection the_global_simplicial_connection (end)

    \subsection{The first simplicial Atiyah class} % (fold)
    \label{sub:the-first-simplicial-atiyah-class}

        Working over $U_{\alpha_0\cdots\alpha_p}$ gives us an expression\footnote{Recall that $\d\omega_{\alpha\beta}=-\omega_{\alpha\beta}^2$.} for the first simplicial Atiyah class:
        \begin{equation}\label{equation:first-simplicial-atiyah-class}
            \underline{\at}^1_\E(U_{\alpha_0\cdots\alpha_p}) = \Bigg\{\sum_{i=1}^p\d t_i\otimes\omega_{\alpha_0\alpha_i} - \sum_{i=1}^pt_i\omega_{\alpha_0\alpha_i}^2 + \sum_{i,j=1}^p t_j\omega_{\alpha_0\alpha_j}t_i\omega_{\alpha_0\alpha_i}\Bigg\}_{p\in\mathbb{N}}.
        \end{equation}

        \Cref{equation:fibre-integration-decomposition-1} tells us that the fibre integral of $\underline{\at}_\E^1$ depends only on the $(2,0)$, $(1,1)$, and $(0,2)$ parts.
        Looking at \cref{equation:first-simplicial-atiyah-class} we see that there is no $(2,0)$ part, and so
        \begin{align*}
            \int_{|\Delta|}\underline{\at}_\E^1(U_{\alpha_0\cdots\alpha_p}) &= \int_{|\Delta_1|}\sum_{i=1}^{p=1}\d t_i\otimes\omega_{\alpha_0\alpha_i} - \int_{|\Delta_0|}\sum_{i=1}^{p=0}t_i\omega_{\alpha_0\alpha_i}^2 + \int_{|\Delta_0|}\sum_{i,j=1}^{p=0} t_j\omega_{\alpha_0\alpha_j}t_i\omega_{\alpha_0\alpha_i}\\
            &= \int_{|\Delta_1|}\d t_1\otimes\omega_{\alpha_0\alpha_1}.
        \end{align*}

        \begin{note}
            Although the $(0,2k)$ part of $\underline{\at}_\E^k$ is generally non-zero, when we fibre integrate we only look at it on the $0$-simplex, and there it \emph{is} zero (since both sums disappear).
        \end{note}

        Continuing our calculation gives the same result as in \cref{sub:the_first_atiyah_class}:
        \begin{equation}\label{equation:first-simplicial-atiyah-class-explicit}
            \tr\int_{|\Delta|}\underline{\at}_\E^1(U_{\alpha_0\cdots\alpha_p}) = \tr\int_0^1\omega_{\alpha_0\alpha_1} \d t_1 = \underbrace{\tr(\omega_{\alpha_0\alpha_1})}_{p=1}.
        \end{equation}

        \begin{note}
            We indicate that the result in \cref{equation:first-simplicial-atiyah-class-explicit} `lives in' $p=1$ to mean that the result is a Čech $1$-cocycle.
            Generally, as in the manual construction, the (fibre integral of the) $k$-th simplicial Atiyah class will have terms in $\cech^{k-i}(\Omega_X^{k+i})$ for $i=0,\ldots,k-1$.
            This is just notational laziness: we write sums to mean direct sums.
        \end{note}

    % subsection the-first-simplicial-atiyah-class (end)

    \subsection{The second simplicial Atiyah class} % (fold)
    \label{sub:the_second_simplicial_atiyah_class}

        We know that
        \begin{equation*}
            \underline{\at}_\E^2 = \Bigg\{\Bigg(\sum_{i=1}^p\d t_i\otimes\omega_{\alpha_0\alpha_i} - \sum_{i=1}^pt_i\omega_{\alpha_0\alpha_i}^2 + \sum_{i,j=1}^p t_j\omega_{\alpha_0\alpha_j}t_i\omega_{\alpha_0\alpha_i}\Bigg)^2\Bigg\}_{p\in\mathbb{N}}
        \end{equation*}
        but also that the only parts that will be non-zero after fibre integration are the $(2,2)$ parts on the $2$-simplex, and the $(1,3)$ parts on the $1$-simplex.
        The only $(2,2)$ part comes from the $(\d\overline\omega)^2$ term, and, recalling \cref{sec:the_dupont_isomorphism} and remembering the sign $\epsilon_2=-1$, this is
        \begin{align*}
            -\int_{|\Delta_2|}\sum_{i,j=1}^p (\d t_j\otimes\omega_{\alpha_0\alpha_j}) \wedge (\d t_i\otimes\omega_{\alpha_0\alpha_i}) = &\int_{|\Delta_2|}\sum_{i,j=1}^{p=2} \d t_j\d t_i\otimes\omega_{\alpha_0\alpha_j}\omega_{\alpha_0\alpha_i}\\
            = &\int_{|\Delta_2|}\Big((\d t_1)^2\otimes\omega_{\alpha_0\alpha_1}^2 + \d t_1\d t_2\otimes\omega_{\alpha_0\alpha_1}\omega_{\alpha_0\alpha_2}\\
            &\qquad\quad+ \d t_2\d t_1\otimes\omega_{\alpha_0\alpha_2}\omega_{\alpha_0\alpha_1} + (\d t_2)^2\otimes\omega_{\alpha_0\alpha_2}^2\Big)\\
            = &\int_{|\Delta_2|} \d t_1\d t_2\otimes\Big(\omega_{\alpha_0\alpha_1}\omega_{\alpha_0\alpha_2} - \omega_{\alpha_0\alpha_2}\omega_{\alpha_0\alpha_1}\Big)\\
            = &\int_0^1\int_0^{1-t_2} \d t_1\d t_2\otimes\Big(\omega_{\alpha_0\alpha_1}\omega_{\alpha_0\alpha_2} - \omega_{\alpha_0\alpha_2}\omega_{\alpha_0\alpha_1}\Big)\\
            = &\,\,\frac12\Big(\omega_{\alpha_0\alpha_1}\omega_{\alpha_0\alpha_2} - \omega_{\alpha_0\alpha_2}\omega_{\alpha_0\alpha_1}\Big)\\
            \circeq&\,\, \frac{1}{2}\cdot2\cdot\omega_{\alpha_0\alpha_1}\omega_{\alpha_0\alpha_2}\\
            \circeq&\,\, \omega_{\alpha_0\alpha_1}(\omega_{\alpha_0\alpha_1}+\omega_{\alpha_1\alpha_2})\\
            \circeq&\,\, \omega_{\alpha_0\alpha_1}\omega_{\alpha_1\alpha_2}.
        \end{align*}
        So far, this agrees with the result found in \cref{sub:the_second_atiyah_class}:
        \begin{equation}
            \tr\int_{|\Delta|}\underline{\at}_\E^2(U_{\alpha_0\cdots\alpha_p}) = \underbrace{\tr(\omega_{\alpha_0\alpha_1}\omega_{\alpha_1\alpha_2})}_{p=2} + \underbrace{?}_{p=1}
        \end{equation}

        Now, remembering again the sign \mbox{$\epsilon_2=-1$}, we calculate the $(1,3)$ part on the $1$-simplex:
        \begin{align*}
            &\sum_{i,j=1}^p \d t_j\otimes\omega_{\alpha_0\alpha_j}\wedge t_i\omega_{\alpha_0\alpha_i}^2 + \sum_{i,j=1}^pt_j\omega_{\alpha_0\alpha_j}^2\wedge(\d t_i\otimes\omega_{\alpha_0\alpha_i})\\
            - &\sum_{i,j,k=1}^p (\d t_k\otimes\omega_{\alpha_0\alpha_k})\wedge t_j\omega_{\alpha_0\alpha_j} t_i\omega_{\alpha_0\alpha_i} - \sum_{i,j,k=1}^p t_k\omega_{\alpha_0\alpha_k} t_j\omega_{\alpha_0\alpha_j}\wedge(\d t_i\otimes\omega_{\alpha_0\alpha_i})\\
            &\overset{\int_{|\Delta|}}{\longmapsto} \int_{|\Delta_1|} 2\omega_{\alpha_0\alpha_1}^3 (t_1\d t_1 - t_1^2\d t_1)\\
            &\qquad\quad= \frac13\omega_{\alpha_0\alpha_1}^3.
        \end{align*}
    Thus, exactly as in \cref{sub:the_second_atiyah_class}, we see that
        \begin{equation}
            \tr\int_{|\Delta|}\underline{\at}_\E^2(U_{\alpha_0\cdots\alpha_p}) = \underbrace{\tr(\omega_{\alpha_0\alpha_1}\omega_{\alpha_1\alpha_2})}_{p=2} + \underbrace{\frac13\tr(\omega_{\alpha_0\alpha_1}^3)}_{p=1}.
        \end{equation}

    % subsection the_second_simplicial_atiyah_class (end)

    \subsection{The third simplicial Atiyah class} % (fold)
    \label{sub:the_third_simplicial_atiyah_class}

        There is a subtlety in the calculations when we reach the third simplicial Atiyah class, due to our choice of conventions for Čech cocycles.
        We don't assume skew-symmetry of cocycles (i.e. that exchanging two indices changes sign), but skew-symmetrisation of cocycles is a quasi-isomorphism, and so doesn't change the cohomology class\footnote{All this says is that you have (at least) two models of Čech cocycles that are quasi-isomorphic: cocycles with arbitrarily-ordered indices, and cocycles with arbitrarily-ordered but skew-symmetric indices.}.
        If we had worked with skew-symmetric Čech cocycles from the start then this calculation would appear in some sense more natural, but we didn't.

        We write $\mu_i$ to mean $\omega_{\alpha_0\alpha_i}$.
        Then, as before, we know that the $(3,3)$ part of $\int_{|\Delta|}\underline{\at}_\E^3$ is
        \begin{align*}
            -\int_{|\Delta_3|}\sum_{i,j,k=1}^p (\d t_k\otimes\mu_k) \wedge (\d t_j\otimes\mu_j) &\wedge (\d t_i\otimes\mu_i) = \int_{|\Delta_3|}\sum_{i,j,k=1}^{p=3} \d t_k\d t_j\d t_i\otimes\mu_k\mu_j\mu_i\\
            &= \int_{|\Delta_3|}\sum_{\sigma\in S_3}\sgn(\sigma)\,\d t_1\d t_2\d t_3\otimes\mu_{\sigma(1)}\mu_{\sigma(2)}\mu_{\sigma(3)}\\
            &= \frac16 \sum_{\sigma\in S_3}\sgn(\sigma)\,\mu_{\sigma(1)}\mu_{\sigma(2)}\mu_{\sigma(3)}\\
            &\circeq \frac12(\mu_1\mu_2\mu_3 - \mu_1\mu_3\mu_2)\\
            &\circeq \frac12\Big(\omega_{\alpha_0\alpha_1}(\omega_{\alpha_0\alpha_1}+\omega_{\alpha_1\alpha_2})(\omega_{\alpha_0\alpha_1}+\omega_{\alpha_1\alpha_2}+\omega_{\alpha_2\alpha_3})\\
            &\quad -\omega_{\alpha_0\alpha_1}(\omega_{\alpha_0\alpha_1}+\omega_{\alpha_1\alpha_2}+\omega_{\alpha_2\alpha_3})(\omega_{\alpha_0\alpha_1}+\omega_{\alpha_1\alpha_2})\Big)\\
            &\circeq \frac12\Big(\omega_{\alpha_0\alpha_1}\omega_{\alpha_1\alpha_2}\omega_{\alpha_2\alpha_3} - \omega_{\alpha_0\alpha_1}\omega_{\alpha_2\alpha_3}\omega_{\alpha_1\alpha_2}\Big).
        \end{align*}
        But note that both of the terms in this last expression skew-symmetrise to the same thing:
        \[
            \sum_{\sigma\in S_4}\sgn(\sigma)\, \omega_{\alpha_{\sigma(0)}\alpha_{\sigma(1)}}\omega_{\alpha_{\sigma(1)}\alpha_{\sigma(2)}}\omega_{\alpha_{\sigma(2)}\alpha_{\sigma(3)}} = -\sum_{\sigma\in S_4}\sgn(\sigma)\, \omega_{\alpha_{\sigma(0)}\alpha_{\sigma(1)}}\omega_{\alpha_{\sigma(2)}\alpha_{\sigma(3)}}\omega_{\alpha_{\sigma(1)}\alpha_{\sigma(2)}}.
        \]
        Thus, writing $\ss_p$ to mean the skew-symmetrisation of a Čech $p$-cocycle, we have shown that
        \[
            \ss_3\tr\int_{|\Delta_3|}\underline{\at}_\E^3 = \ss_3\tr(\at_\E^3)
        \]
        whence they both represent the same class in cohomology.

        \bigskip

        The $(2,4)$ term comes from\footnote{Not forgetting that $\epsilon_3=-1$.}
        \begin{equation}\label{equation:2-4-term-in-x-and-y}
            X^2Y + XYX + YX^2 - X^2Z - XZX - ZX^2
        \end{equation}
        where
        \begin{equation*}
            X = \sum_{i=1}^2\d t_i\otimes\mu_i\qquad
            Y = \sum_{i=1}^2 t_i\mu_i^2\qquad
            Z = \sum_{i,j=1}^2 t_it_j\mu_i\mu_j.
        \end{equation*}

        Using that $\int_{|\Delta_2|}\d t_1\d t_2=\int_0^1\int_0^{1-t_2}\d t_1\d t_2$ we can calculate the following:
        \begin{enumerate}[(i)]
            \item $\int_{|\Delta_2|}t_1\d t_1\d t_2=\int_{|\Delta_2|}t_2\d t_1\d t_2=\frac16$;
            \item $\int_{|\Delta_2|}t_1^2\d t_1\d t_2=\int_{|\Delta_2|}t_2^2\d t_1\d t_2=\frac{1}{12}$;
            \item $\int_{|\Delta_2|}t_1t_2\d t_1\d t_2=\frac{1}{24}$.
        \end{enumerate}
        Then we can integrate \cref{equation:2-4-term-in-x-and-y}:
        \begin{align*}
            \int_{|\Delta_2|}X^2(Z-Y) + X(Z-Y)X + (Z-Y)X^2 &= \frac12\mu_1^3\mu_2 + \frac12\mu_1\mu_2^3 - \frac14\mu_1\mu_2\mu_1\mu_2\\
            &= \frac14(\omega_{\alpha_0\alpha_1}\omega_{\alpha_1\alpha_2})^2 + \frac12(\omega_{\alpha_0\alpha_1}^3\omega_{\alpha_1\alpha_2} + \omega_{\alpha_0\alpha_1}\omega_{\alpha_1\alpha_2}^3)
        \end{align*}
        Comparing this with \cref{equation:2-4-term-manual-construction}, we have the same, save a missing $\frac12\omega_{\alpha_0\alpha_1}^2\omega_{\alpha_1\alpha_2}^2$ term.
        But this skew-symmetrises to zero, since it is invariant under the permutation that swaps $0$ and $2$.
        Thus the $(2,4)$ terms agree in cohomology.

        \bigskip

        Finally, the $(1,5)$ term is
        \begin{align*}
            -\int_{|\Delta_1|}(3t_1^2+ 3t_1^4 - 6t_1^3)\d t_1\otimes\mu_1^5 = &-\int_0^1 (3t_1^2+ 3t_1^4 - 6t_1^3)\d t_1\otimes\mu_1^5\\
            =&-\left(1+\frac35-\frac32\right)\mu_1^5 = -\frac{1}{10}\omega_{\alpha_0\alpha_1}^5
        \end{align*}
        which agrees exactly with \cref{equation:1-5-term-manual-construction}.

    % subsection the_third_simplicial_atiyah_class (end)

    \subsection{General results} % (fold)
    \label{sub:general_results}

        It would be reassuring to see that the classes that we obtain through fibre integration really do agree with our manual construction.
        This is the content of the following theorem.

        \begin{theorem}\label{theorem:k-k-term-of-simplicial-atiyah-class}
            The degree-$(k,k)$ term in $\tr\int_{|\Delta|}\underline{\at}_\E^k$ is equivalent to $\tr(\at_\E^k)$.
            That is,
            \[
                \ss_k\left(\tr\int_{|\Delta|}\underline{\at}_\E^k\right)^{(k,k)} = \ss_k\tr(\at_\E^k)
            \]
            where $\ss_k$ denotes the skew-symmetrisation of a Čech $k$-cocycle.
        \end{theorem}
    
        \begin{proof}

            \textbf{First we rewrite the left-hand side.}
            Generalising \cref{sub:the_third_simplicial_atiyah_class}, we can write the term coming from fibre integration as
            \begin{equation*}
                \frac{1}{k!}\sum_{\sigma\in S_{k}}\sgn{(\sigma)}\mu_{\sigma(1)}\cdots\mu_{\sigma(k)} \overset{\ss_k}{\longmapsto} \frac{1}{k!(k+1)!}\sum_{\tau\in S_{k+1}}\sum_{\sigma\in S_{k}}\sgn{(\tau\sigma)}\omega_{\tau(0)\tau\sigma(1)}\cdots\omega_{\tau(0)\tau\sigma(k)}
            \end{equation*}
            where $S_k\leqslant S_{k+1}$ acts on $\{0,1,\ldots,k\}$ by fixing $0$.
            But then, since $\omega(0)=0$, we can rewrite this as
            \begin{equation*}
                \frac{1}{k!(k+1)!}\sum_{\tau\in S_{k+1}}\sum_{\sigma\in S_{k}}\sgn{(\tau\sigma)}\omega_{\tau\sigma(0)\tau\sigma(1)}\cdots\omega_{\tau\sigma(0)\tau\sigma(k)}.
            \end{equation*}
            Now we can use the fact that multiplication by an element of $S_k\leqslant S_{k+1}$ is an automorphism to do a change of variables.
            This gives us
            \begin{equation*}
                \frac{1}{(k+1)!}\sum_{\eta\in S_{k+1}}\sgn{(\eta)}\,\omega_{\eta(0)\eta(1)}\cdots\omega_{\eta(0)\eta(k)}
            \end{equation*}
            which is trivially equal to
            \begin{equation}\label{eq:lhs-kk}
                \frac{1}{(k+1)!}\sum_{\eta\in S_{k+1}}\sgn{(\eta)}\,\prod_{i=1}^k\big(\omega_{\eta(0)\eta(i)}-\omega_{\eta(0)\eta(0)}\big)
            \end{equation}
            where we use that $\omega_{ii}=0$, by definition.

            \medskip
            \textbf{Next we rewrite the right-hand side.}
            The skew-symmetrisation is simply
            \begin{align}\label{eq:rhs-kk}
                \ss_k\tr(\at^k_\E) &= \frac{1}{(k+1)!}\sum_{\eta\in S_{k+1}}\sgn{(\eta)}\,\omega_{\eta(0)\eta(1)}\omega_{\eta(1)\eta(2)}\cdots\omega_{\eta(k-1)\eta(k)}\nonumber\\
                &= \frac{1}{(k+1)!}\sum_{\eta\in S_{k+1}}\sgn{(\eta)}\,\omega_{\eta(0)\eta(1)}(\omega_{\eta(0)\eta(2)}-\omega_{\eta(0)\eta(1)})\cdots(\omega_{\eta(0)\eta(k)}-\omega_{\eta(0)\eta(k-1)})\nonumber\\
                &= \frac{1}{(k+1)!}\sum_{\eta\in S_{k+1}}\sgn{(\eta)}\,\prod_{i=1}^k\big(\omega_{\eta(0)\eta(i)}-\omega_{\eta(0)\eta(i-1)}\big)
            \end{align}
            again using that $\omega_{ii}=0$.

            \medskip
            \textbf{Now we prove equality.}
            Since, \emph{for each fixed $\eta$}, there are no relations satisfied between the $\omega_{\eta(0)\eta(i)}$, we have two polynomials in $(k+1)$ free non-commutative variables, homogeneous of degree $k$.
            To emphasise the fact the following argument is purely abstract, we write $x_i=\omega_{0i}$ and define an action of $S_{k+1}$ on the $x_i$ by $x_{\eta(i)}=\omega_{\eta(0)\eta(i)}$.
            Now, showing that \cref{eq:lhs-kk,eq:rhs-kk} are equal amounts to showing that
            \begin{equation}\label{eq:a-b-kk}
                A := \sum_{\eta\in S_{k+1}}\sgn{(\eta)}\,\prod_{i=1}^k\big(x_{\eta(i)}-x_{\eta(0)}\big) = \sum_{\eta\in S_{k+1}}\sgn{(\eta)}\,\prod_{i=1}^k\big(x_{\eta(i)}-x_{\eta(i-1)}\big) =: B.
            \end{equation}

            Write $E$ to mean the $\mathbb{Z}$-linear span of degree-$k$ monomials in the $(k+1)$ free non-commutative variables $x_i$, and let $\sigma_{p,q}\in S_{k+1}$ be the transposition that swaps $p$ and $q$.
            Then $\sigma_{p,q}$ gives an involution on $E$, and thus $E\cong E_{p,q}(1)\oplus E_{p,q}(-1)$, where $E_{p,q}(\lambda)$ is the eigenspace corresponding to the eigenvalue $\lambda$.

            Let $H_{p,q}$ be the $\mathbb{Z}$-linear subspace of $E$ spanned by monomials that contain at least one $x_p$ or $x_q$.
            This subspace is clearly stable\footnote{If a monomial $X$ contains, say, one $x_p$, then $\sigma_{p,q}X$ contains one $x_q$.} under $\sigma_{p,q}$, and so this space also splits as $H_{p,q}\cong H_{p,q}(1)\oplus H_{p,q}(-1)$.
            Further, we have the obvious inclusion $H_{p,q}(-1)\subseteq E_{p,q}(-1)$, but we see that if $X\in E_{p,q}(-1)$ then, in particular, $X$ must contain at least one\footnote{If not, then the action of $\sigma_{p,q}$ would be trivial and $X$ would lie in $E_{p,q}(1)$.} $x_p$ or $x_q$, so $X\in H_{p,q}$, whence $X\in H_{p,q}(-1)$.
            Thus $E_{p,q}(-1)=H_{p,q}(-1)$.

            The intersection $H$ of the $H_{p,q}$ over all (distinct) pairs $(p,q)\in\{0,\ldots,k\}\times\{0,\ldots,k\}$ is the $\mathbb{Z}$-linear span of all monomials containing all but one of the $x_i$ (and, in particular, containing $k$ distinct $x_i$).
            But since $H_{p,q}(-1)=E_{p,q}(-1)$, we see that the intersection $E(-1)$ of all the $E_{p,q}(-1)$ is equal to $H(-1)$.
            Now both $A$ and $B$ are in $E_{p,q}(-1)$ for all $p,q$ (since $\sgn(\omega_{p,q})=-1$), and so $A,B\in E(-1)=H(-1)$.
            Since the coefficient of, for example, the $x_1\cdots x_k$ term is equal (and \emph{non-zero}) in both $A$ and $B$ (it is $1$, because we have to take $\eta=\id$), it suffices to show that $H(-1)$ is one-dimensional to prove \cref{eq:a-b-kk}.

            So let $X,Y\in H(-1)$ be monomials.
            Then each one contains $k$ distinct $x_i$, and so there exists some (unique) $\sigma\in S_{k+1}$ such that $\sigma X=\pm Y$.
            But, writing $\sigma=\sigma_{p_1,q_1}\cdots\sigma_{p_r,q_r}$, we know that $\sigma X=(-1)^r X$, whence $X=Y$, up to some sign (i.e. up to some scalar in $\mathbb{Z}$).
        \end{proof}

    % subsection general_results (end)

% section simplicial_construction (end)
